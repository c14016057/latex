%=======================02-713 LaTeX template, following the 15-210 template==================
%
% You don't need to use LaTeX or this template, but you must turn your homework in as
% a typeset PDF somehow.
%
% How to use:
%    1. Update your information in section "A" below
%    2. Write your answers in section "B" below. Precede answers for all 
%       parts of a question with the command "\question{n}{desc}" where n is
%       the question number and "desc" is a short, one-line description of 
%       the problem. There is no need to restate the problem.
%    3. If a question has multiple parts, precede the answer to part x with the
%       command "\part{x}".
%    4. If a problem asks you to design an algorithm, use the commands
%       \algorithm, \correctness, \runtime to precede your discussion of the 
%       description of the algorithm, its correctness, and its running time, respectively.
%    5. You can include graphics by using the command \includegraphics{FILENAME}
%
\documentclass[11pt]{article}

\usepackage[UTF8, heading = false, scheme = plain]{ctex} % chinese !!!!
\usepackage{amsmath,amssymb,amsthm}
\usepackage{graphicx}
\usepackage[margin=1in]{geometry}
\usepackage{fancyhdr}
\usepackage{CJK}

\setlength{\parindent}{0pt}
\setlength{\parskip}{5pt plus 1pt}
\setlength{\headheight}{13.6pt}

% define
\newcommand\question[2]{\vspace{.25in}\hrule\textbf{#1: #2}\vspace{.5em}\hrule\vspace{.10in}}
\renewcommand\part[1]{\vspace{.10in}\textbf{#1}}
\newcommand\algorithm{\vspace{.10in}\textbf{Algorithm: }}
\newcommand\correctness{\vspace{.10in}\textbf{Correctness: }}
\newcommand\runtime{\vspace{.10in}\textbf{Running time: }}
\pagestyle{fancyplain}

% header
\lhead{\textbf{\NAME\ (\ANDREWID)}}
\chead{\textbf{HW\HWNUM}}
\rhead{Theory of Computation, \today}


% start content
\begin{document}\raggedright
%Section A==============Change the values below to match your information==================
\newcommand\NAME{Shiang-Yun Yang 楊翔雲}  % your name
\newcommand\ANDREWID{R04922067}     % your andrew id
\newcommand\HWNUM{2}              % the homework number
%Section B==============Put your answers to the questions below here=======================

% no need to restate the problem --- the graders know which problem is which,
% but replacing "The First Problem" with a short phrase will help you remember
% which problem this is when you read over your homeworks to study.


\question{Problem 1}{Assume $L$ is recursive. Prove that $\bar{L}$ is recursively enumerable.}

\part{Answer:} % \algorithm Describe algorithm here

\begin{itemize}

\item 
$L$ is accepted by $M$, and due to $L$ is recursive, then
\begin{itemize}
	\item If $x \in L$, $M(x) = \text{"yes"}$		\item If $x \notin L$, $M(x) = \text{"no"}$
\end{itemize}

\item 
$\bar{M}$ simulate $M$ and report opposite result,
\begin{itemize}
	\item If $M(x) = \text{"no"}$, $\bar{M}(x) = \text{"yes"}$ and $x \in \bar{L}$ 
	\item If $M(x) = \text{"yes"}$, $\bar{M}(x) = \text{"no"}$ and $x \in \bar{L}$ 
\end{itemize}

\item $\bar{M}$ recognizes $\bar{L}$ and must halt on input. By Proposition 2 (page 52), $\bar{L}$ is recursive, then $\bar{L}$ is recursively enumerable.

\end{itemize}


% \correctness This is an argument  that this algorithm returns the correct answer.

% \runtime Describe here, in big-Oh, the running time and your reasoning for it.

% \part{b}

\question{Problem 2}{Let $A$ and $B$ be two sets and $A \subseteq B$. Prove that $|A| \le |B|$.}

\part{Answer:}

\begin{itemize}
	\item If $A \subseteq B$, then $\forall x \in A: x \in B$, and
	\item Define $|A| \le |B|$ if there is a one-to-one correspondence between $A$ and a subset of B's. (page 132)
\end{itemize}

\part{Prove 1:}

\begin{itemize}
	\item Assume if $A \subseteq B$, then $|A| > |B|$.
	\item Because $|A| > |B|$, $\exists x \in A : x \notin B$
	\item Above condition also causes that there is no a one-to-one correspondence between $A$ and a subset of B's, so if $|A| \subseteq B$, then $|A| \le |B|$.
\end{itemize}

\part{Prove 2:}

\begin{itemize}
	\item $P \rightarrow Q \Leftrightarrow \text{not } Q \rightarrow \text{not } P$ 
	\item Assume if $A \subseteq B$, then $|A| \le |B|$.
	\item If $|A| > |B|$, $\exists x \in A : x \notin B$, then $A \nsubseteq B$, prove it.
\end{itemize}

\end{document}
