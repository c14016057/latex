%=======================02-713 LaTeX template, following the 15-210 template==================
%
% You don't need to use LaTeX or this template, but you must turn your homework in as
% a typeset PDF somehow.
%
% How to use:
%    1. Update your information in section "A" below
%    2. Write your answers in section "B" below. Precede answers for all 
%       parts of a question with the command "\question{n}{desc}" where n is
%       the question number and "desc" is a short, one-line description of 
%       the problem. There is no need to restate the problem.
%    3. If a question has multiple parts, precede the answer to part x with the
%       command "\part{x}".
%    4. If a problem asks you to design an algorithm, use the commands
%       \algorithm, \correctness, \runtime to precede your discussion of the 
%       description of the algorithm, its correctness, and its running time, respectively.
%    5. You can include graphics by using the command \includegraphics{FILENAME}
%
\documentclass[11pt]{article}

\usepackage[UTF8, heading = false, scheme = plain]{ctex} % chinese !!!!
\usepackage{amsmath,amssymb,amsthm}
\usepackage{graphicx}
\usepackage[margin=1in]{geometry}
\usepackage{fancyhdr}
\usepackage{CJK}

\setlength{\parindent}{0pt}
\setlength{\parskip}{5pt plus 1pt}
\setlength{\headheight}{13.6pt}

% define
\newcommand\question[2]{\vspace{.25in}\hrule\textbf{#1: #2}\vspace{.5em}\hrule\vspace{.10in}}
\renewcommand\part[1]{\vspace{.10in}\textbf{#1}}
\newcommand\algorithm{\vspace{.10in}\textbf{Algorithm: }}
\newcommand\correctness{\vspace{.10in}\textbf{Correctness: }}
\newcommand\runtime{\vspace{.10in}\textbf{Running time: }}
\pagestyle{fancyplain}

% header
\lhead{\textbf{\NAME\ (\ANDREWID)}}
\chead{\textbf{HW\HWNUM}}
\rhead{Theory of Computation, \today}


% start content
\begin{document}\raggedright
%Section A==============Change the values below to match your information==================
\newcommand\NAME{Shiang-Yun Yang 楊翔雲}  % your name
\newcommand\ANDREWID{R04922067}     % your andrew id
\newcommand\HWNUM{1}              % the homework number
%Section B==============Put your answers to the questions below here=======================

% no need to restate the problem --- the graders know which problem is which,
% but replacing "The First Problem" with a short phrase will help you remember
% which problem this is when you read over your homeworks to study.


\question{Problem 1}{Write a TM program for the p.30 when $\sum = \{ 0, 1, 2, \bigsqcup, \triangleright \}$}

\part{Answer:} % \algorithm Describe algorithm here

\begin{itemize}

\item 
TM accepts the input if there are two consecutive 1's.

\item % description
Assume $M = (K, \sum, \delta, s)$, where $K = \{ s, s_1, h \}$, $\sum = \{ 0, 1, 2, \sqcup, \triangleright \}$, and

\end{itemize}

\begin{center}
	\begin{tabular}{|c|c|c|}
	\hline
	$p \in K$ & $ \sigma \in \sum $ & $\delta(p, \sigma)$  \\		
	\hline
	$s$ & $\triangleright$ & $(s, \triangleright, \rightarrow)$  \\
	\hline
	$s$ & $0$ & $(s, 0, \triangleright)$ \\
	\hline
	$s$ & $1$ & $(s_1, 1, \triangleright)$ \\
	\hline
	$s$ & $2$ & $(s, 2, \triangleright)$ \\
	\hline
	$s_1$ & $0$ & $(s, 0, \triangleright)$ \\
	\hline
	$s_1$ & $1$ & $(\text{"yes"}, 1, -)$ \\
	\hline
	$s_1$ & $2$ & $(s, 2, \triangleright)$ \\
	\hline
	$s$ & $\sqcup$ & $(\text{"no"}, \sqcup, -)$ \\
	\hline
	$s_1$ & $\sqcup$ & $(\text{"no"}, \sqcup, -)$ \\
	\hline
	\end{tabular}
\end{center}


% \correctness This is an argument  that this algorithm returns the correct answer.

% \runtime Describe here, in big-Oh, the running time and your reasoning for it.

% \part{b}

\question{Problem 2}{Explain why the following TM doesn't not decide the language $L$}

\begin{itemize}
\item
$L$ : polynomials with integer coefficients which have integer roots.
\item
The input represents a polynomial $P(X)$ over variables $x_1, \cdots, x_n$ witch integer coefficients.
\end{itemize}

\algorithm TM
\begin{enumerate}
  \item Examine all possible integer values of $x_1, \cdots, x_n$.
  \item Evaluate the polynomial on all of them.
  \item If any of them evaluates to 0, accept; else reject.
\end{enumerate}

\part{Answer:}

By the definition says $M$ decides $L$, let $M$ be a TM such that for any string x:
\begin{itemize}
  \item[-] If $x \in L$, then $M(x) = \text{"yes"}$.
  \item[-] If $x \notin L$, then $M(x) = \text{"no"}$.
\end{itemize}


We observe that TM will enumerate all the integer $x \in \mathbb{Z}$ in step 1. Because $\mathbb{Z}$ is infinite in size, TM won't halt when the $P(x) \notin L$, i.e.,
\begin{itemize}
  \item[-] If $x \in L$, then $M(x) = \text{"yes"}$.
  \item[-] If $x \notin L$, then $M(x) = \nearrow$.
\end{itemize}

TM doesn't satisfy the definition of $M$ decides $L$, so the TM doesn't not decide the language $L$.

\end{document}
