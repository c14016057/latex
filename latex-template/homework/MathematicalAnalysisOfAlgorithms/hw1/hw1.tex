%=======================02-713 LaTeX template, following the 15-210 template==================
%
% You don't need to use LaTeX or this template, but you must turn your homework in as
% a typeset PDF somehow.
%
% How to use:
%    1. Update your information in section "A" below
%    2. Write your answers in section "B" below. Precede answers for all 
%       parts of a question with the command "\question{n}{desc}" where n is
%       the question number and "desc" is a short, one-line description of 
%       the problem. There is no need to restate the problem.
%    3. If a question has multiple parts, precede the answer to part x with the
%       command "\part{x}".
%    4. If a problem asks you to design an algorithm, use the commands
%       \algorithm, \correctness, \runtime to precede your discussion of the 
%       description of the algorithm, its correctness, and its running time, respectively.
%    5. You can include graphics by using the command \includegraphics{FILENAME}
%
\documentclass[11pt]{article}

\usepackage[UTF8, heading = false, scheme = plain]{ctex} % chinese !!!!
\usepackage{amsmath,amssymb,amsthm}
\usepackage{graphicx}
\usepackage[margin=1in]{geometry}
\usepackage{fancyhdr}
\usepackage{CJK}
\usepackage{listings}

\setCJKmainfont{STHeiti} % chinese type

\setlength{\parindent}{0pt}
\setlength{\parskip}{5pt plus 1pt}
\setlength{\headheight}{13.6pt}

% define
\newcommand\question[2]{\vspace{.25in}\hrule\textbf{#1: #2}\vspace{.5em}\hrule\vspace{.10in}}
\renewcommand\part[1]{\vspace{.10in}\textbf{#1}}
\newcommand\algorithm{\vspace{.10in}\textbf{Algorithm: }}
\newcommand\correctness{\vspace{.10in}\textbf{Correctness: }}
\newcommand\runtime{\vspace{.10in}\textbf{Running time: }}
\pagestyle{fancyplain}

% header
\lhead{\textbf{\NAME\ (\ANDREWID)}}
\chead{\textbf{HW\HWNUM}}
\rhead{演算法數學解析, \today}


% start content
\begin{document}\raggedright
%Section A==============Change the values below to match your information==================
\newcommand\NAME{Shiang-Yun Yang 楊翔雲}  % your name
\newcommand\ANDREWID{R04922067}     % your andrew id
\newcommand\HWNUM{1}              % the homework number
%Section B==============Put your answers to the questions below here=======================

% no need to restate the problem --- the graders know which problem is which,
% but replacing "The First Problem" with a short phrase will help you remember
% which problem this is when you read over your homeworks to study.


\question{Problem 1}{1.9 Sometimes it's possible to use induction backwards, proving thing from $n$ to $n-1$ instead of vice versa! For example, consider the statement
$$P(n): x_1 \hdots x_n \le \left ( \frac{x_1 + \hdots + x_n}{n}\right )^n, \; \text{ if } x_1, \; \hdots x_n \ge 0$$
}

This is true when $n = 2$, since $(x_1 + x_2)^2 - 4 x_1 x_2 = (x_1 - x_2)^2 \ge 0$

\begin{description}
	\item[a] By setting $x_n = (x_1 + \hdots + x_{n-1})/(n-1)$, prove that $P(n)$ implies $P(n-1)$ whenever $n > 1$.
	\item[b] Show that $P(n)$ and $P(2)$ imply $P(2n)$.
	\item[c] Explain why this implies the truth of $P(n)$ for all $n$.
\end{description}


\part{Answer:} 

\begin{description}
	\item[a]
		若 $P(n)$ 成立,令 $x_n = (x_1 + \hdots + x_{n-1})/(n-1)$,則 $P(n-1)$ 成立。
		\begin{equation}
		\begin{split}
			& \Rightarrow x_1 \; x_2 \; \hdots \; x_n \le \left ( \frac{x_1 + \hdots + x_n}{n}\right )^n \\
			& \Rightarrow x_1 \; x_2 \; \hdots \; x_{n-1} \; \frac{x_1 + x_2 + \hdots + x_{n-1}}{n-1} \le \left ( \frac{x_1 + \hdots + x_{n-1}}{n} + \frac{x_n}{n} \right )^n \\
			& \Rightarrow x_1 \; x_2 \; \hdots \; x_{n-1} \; \frac{x_1 + x_2 + \hdots + x_{n-1}}{n-1} \le \left ( \frac{x_1 + \hdots + x_{n-1}}{n} + \frac{x_1 + x_2 + \hdots + x_{n-1}}{n(n-1)} \right )^n \\
			& \Rightarrow x_1 \; x_2 \; \hdots \; x_{n-1} \; \frac{x_1 + x_2 + \hdots + x_{n-1}}{n-1} \le \left ( \frac{x_1 + x_2 + \hdots + x_{n-1}}{n-1} \right )^n \\
			& \Rightarrow x_1 \; x_2 \; \hdots \; x_{n-1} \le \left ( \frac{x_1 + x_2 + \hdots + x_{n-1}}{n-1} \right )^{n-1}
		\end{split}
		\end{equation}
	\item[b]
		若 $P(n)$ 和 $P(2)$ 成立,則 $P(2n)$ 成立。
		\begin{equation}
		\begin{split}
			x_1 \; x_2 \; \hdots \; x_n \le \left ( \frac{x_1 + \hdots + x_n}{n}\right )^n
		\end{split}
		\end{equation}
		\begin{equation}
		\begin{split}
			x_{n+1} \; x_{n+2} \; \hdots \; x_{2n} \le \left ( \frac{x_{n+1} + \hdots + x_{2n}}{n}\right )^n
		\end{split}
		\end{equation}
		由 $P(n)$ 得到上述兩式,兩者合併得到
		\begin{equation}
		\begin{split}
			x_1 \; x_2 \; \hdots \; x_{2n} \le \left [ \left ( \frac{x_1 + \hdots + x_n}{n} \right ) \left ( \frac{x_{n+1} + \hdots + x_{2n}}{n} \right ) \right ]^n
		\end{split}
		\end{equation}
		藉由 $P(2)$ 轉換不等式右側
		\begin{equation}
		\begin{split}
			x_1 \; x_2 \; \hdots \; x_{2n} & \le \left [ \left ( 
				\frac{
					\frac{x_1 + \hdots + x_n}{n} + \frac{x_{n+1} + \hdots + x_{2n}}{n}
					}{2} 
				\right )^2 \right ]^n
		\end{split}
		\end{equation}
		化簡得到
		\begin{equation}
		\begin{split}
			x_1 \; x_2 \; \hdots \; x_{2n} & \le \left ( 
				\frac{
					x_1 + \hdots + x_n + x_{n+1} + \hdots + x_{2n}
					}{2n} 
				\right )^{2n}
		\end{split}
		\end{equation}
	\item[c] 藉由 \textbf{b} 的事實,可以倍增到無限大 $P(2^{m}), \; m \rightarrow \infty$ 皆成立,再從 \textbf{a} 得到 $P(n), \; n \le m$ 皆成立,故 $P(n), \; n > 1$ 恆成立。
\end{description}

% part 2%

\question{Problem 2}{1.16 Use the repertoire method to solve the general four-parameter recurrence
		\begin{equation}
		\begin{split}
			g(1) &= \alpha \\
			g(2n+j) &= 3 g(n) + \gamma n + \beta_j \text{, for } j = 0, 1 \text{ and } n \ge 1
		\end{split}
		\end{equation}
}

\part{Answer:} 

從遞迴式展開中得知,$g(n) = A(n) \alpha + B(n) \beta_0 + C(n) \beta_1 + D(n) \gamma$,利用 repertoire method,反求 $A(n)$, $B(n)$, $C(n)$, $D(n)$。

\begin{enumerate}
	\item Case 1: $(\alpha, \beta_0, \beta_1, \gamma) = (1, 0, 1, -1)$
		\begin{equation}
		\begin{split}
			g(1) &= 1 \\ 
			g(2n+j) &= 3g(n)-n + \beta_j \text{, for } j = 0, 1 \text{ and } n \ge 1
		\end{split}
		\end{equation}
		得到 $g(n) = A(n) + C(n) - D(n) = n$
	\item Case 2: $(\alpha, \beta_0, \beta_1, \gamma) = (1, 0, 0, 0)$
		\begin{equation}
		\begin{split}
			g(1) &= 1 \\ 
			g(2n+j) &= 3g(n) \text{, for } n \ge 1
		\end{split}
		\end{equation}
		得到 $g(n) = A(n) = 3^{\log \left \lfloor \frac{n}{2} \right \rfloor + 1}$
	\item Case 3: $(\alpha, \beta_0, \beta_1, \gamma) = (1, -2, -2, 0)$
		\begin{equation}
		\begin{split}
			g(1) &= 1 \\ 
			g(2n+j) &= 3g(n) - 2\text{, for } n \ge 1
		\end{split}
		\end{equation}
		得到 $g(n) = A(n) - 2 B(n) - 2 C(n) = 1$
	\item Case 4: $(\alpha, \beta_0, \beta_1, \gamma) = (1, 0, 1, 0)$
		\begin{equation}
		\begin{split}
			g(1) &= 1 \\ 
			g(2n+j) &= 3g(n) + \beta_j \text{, for } j = 0, 1 \text{ and }n \ge 1
		\end{split}
		\end{equation}
		得到 $g(n) = A(n) + C(n) = \textit{atoi}(\textit{itoa}(n, base_2), base_3)$
\end{enumerate}

解聯立方程得到 

\begin{itemize}
	\item $A(n) = 3^{\log \left \lfloor \frac{n}{2} \right \rfloor + 1}$
	\item $C(n) = \textit{atoi}(\textit{itoa}(n, base_2), base_3) - 3^{\log \left \lfloor \frac{n}{2} \right \rfloor + 1}$
	\item $D(n) = \textit{atoi}(\textit{itoa}(n, base_2), base_3) - n$
	\item $B(n) = (A(n) - 2 C(n) - 1) / 2$
\end{itemize}

% part 3%

\question{Problem 3}{2.11 The general rule (2.56) for summation by pars is equivalent to
		$$
		\sum_{0 \le k < n} (a_{k+1} - a_k) b_k = a_n b_n - a_0 b_0 
												- \sum_{0 \le k < n} a_{k+1} (b_{k+1} - b_k)\text{, for } n \ge 0
		$$
		Prove this formula directly by using the distributive, associative, and commutative laws.
}

\part{Answer:} 

\begin{align*}
	\sum_{0 \le k < n} (a_{k+1} - a_k) b_k &= \sum_{0 \le k < n} (a_{k+1} b_k - a_k b_k) 
					&& \text{分配律} \\
			&= \sum_{0 \le k < n} a_{k+1} b_k - \sum_{0 \le k < n} a_k b_k 
					&& \text{分配律} \\
			&= \sum_{0 \le k < n} a_{k+1} b_k - (a_0 b_0 + \sum_{1 \le k < n} a_k b_k) 
					&& \text{拆首項} \\
			&= \sum_{0 \le k < n} a_{k+1} b_k - (a_0 b_0 + \sum_{0 \le k < n-1} a_{k+1} b_{k+1}) 
					&& \text{變量修改} \\
			&= - a_0 b_0 + \sum_{0 \le k < n} a_{k+1} b_k - \sum_{0 \le k < n-1} a_{k+1} b_{k+1} 
					&& \text{} \\
			&= - a_0 b_0 + \sum_{0 \le k < n} a_{k+1} b_k - (\sum_{0 \le k < n} a_{k+1} b_{k+1} - a_n b_n) 
					&& \text{補項調整} \\
			&= a_n b_n - a_0 b_0 + \left ( \sum_{0 \le k < n} a_{k+1} b_k - \sum_{0 \le k < n} a_{k+1} b_{k+1} \right ) 
					&& \text{結合律} \\
			&= a_n b_n - a_0 b_0 - \sum_{0 \le k < n} a_{k+1} (b_{k+1} - b_k)
					&& \text{等價得證} \\
\end{align*}

\question{Problem 4}{2.22 Prove \textit{Lagrange's identity} (without using induction):
	$$
		\sum_{1 \le j < k \le n} (a_j b_k - a_k b_j)^2 = 
			\left ( \sum_{k=1}^{n} a_k^2 \right )
			\left ( \sum_{k=1}^{n} b_k^2 \right )
		- \left ( \sum_{k=1}^{n} a_k b_k \right )^2
	$$
	This, incidentally, implies \textit{Cauchy's inequality},
	$$
		\left ( \sum_{k = 1}^{n} a_k b_k \right )^2 \le
			\left ( \sum_{k=1}^{n} a_k^2 \right )
			\left ( \sum_{k=1}^{n} b_k^2 \right )
	$$
}

\part{Answer:} 

定義 $S(n)$ 如下:

\begin{equation}
\begin{split}
	S(n) &= \sum_{1 \le j < k \le n} (a_j b_k - a_k b_j)^2 \\
\end{split}
\end{equation}

交換 $j$, $k$。

\begin{equation}
\begin{split}
	S(n) &= \sum_{1 \le k < j \le n} (a_k b_j - a_j b_k)^2 
\end{split}
\end{equation}

得到 $(1 \le j < k \le n) + (1 \le k < j \le n) = (1 \le j, k \le n) - (1 \le j = k \le n)$,將兩式相加得到 $2S(n) = \sum_{1 \le j, k \le n} (a_j b_k - a_k b_j)^2 - \sum_{1 \le j = k \le n} (a_j b_k - a_k b_j)^2$,

\begin{align*}
	2S(n) &= \sum_{1 \le j, k \le n} (a_j b_k - a_k b_j)^2 - \sum_{1 \le j = k \le n} (a_j b_k - a_k b_j)^2 
			\\
		  &= \sum_{1 \le j, k \le n} (a_j b_k - a_k b_j)^2 
		  	&& \because \sum_{1 \le j = k \le n} (a_j b_k - a_k b_j)^2 = 0 \\
		  &= \sum_{1 \le j, k \le n} (a_j b_k)^2 - 2 a_j a_k b_j b_k + (a_k b_j)^2
		  	&& \because \text{分配律}\\
		  &= \sum_{1 \le j, k \le n} (a_j b_k)^2 - \sum_{1 \le j, k \le n} 2 a_j a_k b_j b_k 
		  	+ \sum_{1 \le j, k \le n} (a_k b_j)^2 \\
		  &= 2\sum_{1 \le j, k \le n} (a_j b_k)^2 - \sum_{1 \le j, k \le n} 2 a_j a_k b_j b_k 
\end{align*}

分開計算兩式

\begin{align*}
	\sum_{1 \le j, k \le n} (a_j b_k)^2 &= \sum_{1 \le j \le n} \sum_{1 \le k \le n} a_j^2 b_k^2 
			&& \text{兩變數獨立} \\
			&= \left ( \sum_{k=1}^{n} a_k^2 \right ) \left ( \sum_{k=1}^{n} b_k^2 \right )
\end{align*}

\begin{align*}
	\sum_{1 \le j, k \le n} 2 a_j a_k b_j b_k &= 2 \sum_{1 \le j, k \le n} a_j a_k b_j b_k \\
		&= 2 \sum_{1 \le j \le n} a_j b_j \sum_{1 \le k \le n} a_k b_k \\
		&= 2 \left ( \sum_{1 \le j \le n} a_j b_j \right ) \left ( \sum_{1 \le k \le n} a_k b_k \right ) \\
		&= 2 \left ( \sum_{1 \le j \le n} a_j b_j \right )^2
\end{align*}

兩式代入得到

\begin{align*}
	2S(n) &= 2 \left ( \sum_{k=1}^{n} a_k^2 \right ) \left ( \sum_{k=1}^{n} b_k^2 \right )
				- 2 \left ( \sum_{1 \le j \le n} a_j b_j \right )^2 \\
	 S(n) &= \left ( \sum_{k=1}^{n} a_k^2 \right ) \left ( \sum_{k=1}^{n} b_k^2 \right )
				- \left ( \sum_{1 \le j \le n} a_j b_j \right )^2 \\
	\sum_{1 \le j < k \le n} (a_j b_k - a_k b_j)^2 &= 
			\left ( \sum_{k=1}^{n} a_k^2 \right )
			\left ( \sum_{k=1}^{n} b_k^2 \right )
		- \left ( \sum_{k=1}^{n} a_k b_k \right )^2	
	&& \text{得證} \textit{Lagrange's identity} \\
\end{align*}

又因為 $\sum_{1 \le j < k \le n} (a_j b_k - a_k b_j)^2$ 每一項非負,故

\begin{align*}
\left ( \sum_{k=1}^{n} a_k^2 \right )
\left ( \sum_{k=1}^{n} b_k^2 \right ) - \left ( \sum_{k=1}^{n} a_k b_k \right )^2 & \ge 0 \\
\left ( \sum_{k = 1}^{n} a_k b_k \right )^2 & \le
			\left ( \sum_{k=1}^{n} a_k^2 \right )
			\left ( \sum_{k=1}^{n} b_k^2 \right ) 
		&& \text{得證} \textit{Cauchy's inequality} \\
\end{align*}

\question{Problem 5}{ Prove or disprove that the \textit{Knuth Sequence} defined by 
	\begin{align*}
		K(0) &= 1 \\ 
		K(n+1) &= 1 + \text{min}(2K(\left \lfloor \frac{n}{2} \right \rfloor), 3K(\left \lfloor \frac{n}{3} \right \rfloor))	&& \text{for } n \ge 0
	\end{align*}
	, has the property that $K(n) \ge n$ , for $n \ge 0$. \\
	(The sequence begins 1, 3, 3, 4, 7, 7, 7, 9, 9, 10, 13, $\hdots$)
}

\part{Answer:}

若 $K(n) > n$ , for $n \ge 0$ 成立,則 $K(n) \ge n$ , for $n \ge 0$ 成立。證明 $K(n) > n$ , for $n \ge 0$:

\begin{itemize}
	\item $K(0) = 1 > 0$ 成立。
	\item 若 $K(i) > i \text{, for } i < n$ 成立,則 $K(n)$ 成立。
		由於要取整數,分成 6 種 Case 討論。
		\begin{itemize}
			\item Case 1: $n = 6m$ 時,
				\begin{align*}
					K(n) = 1 + \text{min}(2K(\frac{n-2}{2}), 3K(\frac{n-3}{3}))
				\end{align*}
			\item Case 2: $n = 6m+1$ 時,
				\begin{align*}
					K(n) = 1 + \text{min}(2K(\frac{n-1}{2}), 3K(\frac{n-1}{3}))
				\end{align*}
			\item Case 3: $n = 6m+2$ 時,
				\begin{align*}
					K(n) = 1 + \text{min}(2K(\frac{n-2}{2}), 3K(\frac{n-2}{3}))
				\end{align*}
			\item Case 4: $n = 6m+3$ 時,
				\begin{align*}
					K(n) = 1 + \text{min}(2K(\frac{n-1}{2}), 3K(\frac{n-3}{3}))
				\end{align*}
			\item Case 5: $n = 6m+4$ 時,
				\begin{align*}
					K(n) = 1 + \text{min}(2K(\frac{n-2}{2}), 3K(\frac{n-1}{3}))
				\end{align*}
			\item Case 6: $n = 6m+5$ 時,
				\begin{align*}
					K(n) = 1 + \text{min}(2K(\frac{n-1}{2}), 3K(\frac{n-2}{3}))
				\end{align*}
		\end{itemize}
	每一項都要大於 $n$,才能證明在任一種情況皆大於 $n$,共計 5 項計算。
	\begin{itemize}
		\item
			\begin{align*}
				1 + 2K(\frac{n-1}{2}) & \ge 1 + 2 \left ( \frac{n-1}{2} + \epsilon \right )
					&& K(\frac{n-1}{2}) > \frac{n-1}{2}, \epsilon \ge 1\\
					& \ge 1 + n - 1 + 2 \epsilon \\
					& \ge n + 2 \epsilon > n
			\end{align*}
		\item 
			\begin{align*}
				1 + 2K(\frac{n-2}{2}) & \ge 1 + 2 \left ( \frac{n-2}{2} + \epsilon \right )
					&& K(\frac{n-2}{2}) > \frac{n-2}{2}, \epsilon \ge 1\\
					& \ge 1 + n - 2 + 2 \epsilon \\
					& \ge n - 1 + 2 \epsilon \\
					& \ge n + \epsilon > n
			\end{align*}
		\item
			\begin{align*}
				1 + 3K(\frac{n-1}{3}) & \ge 1 + 3 \left ( \frac{n-1}{3} + \epsilon \right )
					&& K(\frac{n-1}{3}) > \frac{n-1}{3}, \epsilon \ge 1\\
					& \ge 1 + n - 1 + 3 \epsilon \\
					& \ge n + 3 \epsilon > n\\
			\end{align*}
		\item
			\begin{align*}
				1 + 3K(\frac{n-2}{3}) & \ge 1 + 3 \left ( \frac{n-2}{3} + \epsilon \right )
					&& K(\frac{n-2}{3}) > \frac{n-2}{3}, \epsilon \ge 1\\
					& \ge 1 + n - 2 + 3 \epsilon \\
					& \ge n - 1 + 3 \epsilon \\
					& \ge n + 2 \epsilon > n
			\end{align*}
		\item
			\begin{align*}
				1 + 3K(\frac{n-3}{3}) & \ge 1 + 3 \left ( \frac{n-3}{3} + \epsilon \right )
					&& K(\frac{n-3}{2}) > \frac{n-3}{2}, \epsilon \ge 1\\
					& \ge 1 + n - 3 + 3 \epsilon \\
					& \ge n - 2 + 3 \epsilon \\
					& \ge n + \epsilon > n
			\end{align*}
	\end{itemize}
	由於上述接大於 $n$,不管是哪一種都成立,得證 $K(n) > n$ 成立。
	\item 由數學歸納法得知,$K(n) > n$ , for $n \ge 0$ 恆成立。故 $K(n) \ge n$ , for $n \ge 0$ 成立。
\end{itemize}

\question{Problem 5}{Consider the series of fractions
	$$
		\frac{1}{2}, \; 
		\frac{1/2}{3/4}, \; 
		\frac{\frac{1}{2}/\frac{3}{4}}{\frac{5}{6}/\frac{7}{8}}, \; 
		\frac{\frac{1/2}{3/4}/\frac{5/6}{7/8}}{\frac{9/10}{11/12}/\frac{13/14}{15/16}}, \;
		\hdots
	$$
	Suppose that each fraction is simplified to be a fraction of two products of integers 
	(for example, the third is $\frac{1 \cdot 4 \cdot 6 \cdot 7}{2 \cdot 3 \cdot 5 \cdot 8}$.)
	Prove that, for the $n^{\text{th}}$ fraction, the sum of the $k^{\text{th}}$ powers of the 
	numbers in the numerator equals the sum of the $k^{\text{th}}$ powers of the numbers
	in the denominator for $0 \le k < n$. \\
	(For example, $1^2 + 4^2 + 6^2 + 7^2 = 2^2 + 3^2 + 5^2 + 8^2$.)
}

\part{Answer:}

化簡每一項,從數字由小到大描繪分佈看來,如下所示

\begin{align*}
	S(1) &= | \\
	S(2) &= \sqcup \\
	S(3) &= \sqcup \sqcap \\
	S(4) &= \sqcup \sqcap \sqcap \sqcup \\
	S(5) &= \sqcup \sqcap \sqcap \sqcup \sqcap \sqcup \sqcup \sqcap \\
		 & \vdots \\
	S(n) &= S(n-1) \text{flip}(S(n-1)) 
\end{align*}

當 $n > 1$,令 $m = 2^{n-1}$,最後展開的結果如下:

\begin{align*}
	S(n) &= \frac{a_1 \cdot a_2 \hdots a_m}{b_1 \cdot b_2 \hdots b_m} \\ 
	S(n+1) &= \frac{a_1 \cdot a_2 \hdots a_m}{b_1 \cdot b_2 \hdots b_m} \cdot 
			\frac{c_1 \cdot c_2 \hdots c_m}{d_1 \cdot d_2 \hdots d_m}
\end{align*}

滿足下述關係

\begin{itemize}
	\item $(a_1, a_2, \hdots, a_m) = (d_1, d_2, \hdots, d_m) - (m, m, \hdots, m)$.
	\item $(b_1, b_2, \hdots, b_m) = (c_1, c_2, \hdots, c_m) - (m, m, \hdots, m)$.
\end{itemize}

若對於前 $n$ 滿足 $\sum_{i=1}^{m} a_i^k = \sum_{i=1}^{m} b_i^k \text{, for all } 0 \le k < n$,則對於第 $n+1$ 項也會滿足。

\begin{itemize}
	\item 已知 $S(1)$ 代入 $k = 0$ 滿足等式和 $S(2)$ 代入 $k = 0, 1$ 也滿足等式。
	\item 若 $S(i), i \le n$ 成立,則 $S(n+1)$ 也成立。
		\begin{itemize}
			\item 對於任意 $0 \le k < n+1$,滿足 
				\begin{equation}
				\begin{split}
					\sum_{i=1}^{m} a_i^k + \sum_{i=1}^{m} c_i^k &= 
						\sum_{i=1}^{m} b_i^k + \sum_{i=1}^{m} d_i^k \\ 
				\end{split}
				\end{equation}
			\item 已知關係如下
				\begin{equation}
				\begin{split}				
					\sum_{i=1}^{m} a_i^k + \sum_{i=1}^{m} c_i^k &=
						\sum_{i=1}^{m} a_i^k + \sum_{i=1}^{m} (b_i + m)^k \\ 
					\sum_{i=1}^{m} b_i^k + \sum_{i=1}^{m} d_i^k &=
						\sum_{i=1}^{m} b_i^k + \sum_{i=1}^{m} (a_i + m)^k \\ 
				\end{split}
				\end{equation}
			\item 將右式根據二項式展開
				\begin{equation}
				\begin{split}
					\sum_{i=1}^{m} a_i^k + \sum_{i=1}^{m} c_i^k &=
						\sum_{i=1}^{m} a_i^k + \sum_{i=1}^{m} \left (
							b_i^k + \binom{k}{1} b_i^{k-1} m + 
									\binom{k}{2} b_i^{k-2} m^2 + \cdots
									\binom{k}{k} b_i^{0} m^k
							\right ) \\
						&= \sum_{i=1}^{m} a_i^k + \sum_{i=1}^{m} b_i^k + \sum_{i=1}^{m} \left (
									\binom{k}{2} b_i^{k-2} m^2 + \cdots
									\binom{k}{k} b_i^{0} m^k
							\right )
				\end{split}
				\end{equation}
				
				\begin{equation}
				\begin{split}
					\sum_{i=1}^{m} b_i^k + \sum_{i=1}^{m} a_i^k &=
						\sum_{i=1}^{m} b_i^k + \sum_{i=1}^{m} \left (
							a_i^k + \binom{k}{1} a_i^{k-1} m + 
									\binom{k}{2} a_i^{k-2} m^2 + \cdots
									\binom{k}{k} a_i^{0} m^k
							\right ) \\
						&= \sum_{i=1}^{m} b_i^k + \sum_{i=1}^{m} a_i^k + \sum_{i=1}^{m} \left (
									\binom{k}{2} a_i^{k-2} m^2 + \cdots
									\binom{k}{k} a_i^{0} m^k
							\right )
				\end{split}
				\end{equation}
			\item 已知 $S(n)$ 成立,故 $\sum_{i=1}^{m} a_i^k = \sum_{i=1}^{m} b_i^k \text{, for } 0 \le k < n$,代回上述展開等式得到
				\begin{equation}
				\begin{split}
					\because &\sum_{i=1}^{m} a_i^k = 
								\sum_{i=1}^{m} b_i^k \text{, for } 0 \le k < n \\
					\therefore &\sum_{i=1}^{m} \left (
									\binom{k}{2} a_i^{k-2} m^2 + \cdots
									\binom{k}{k} a_i^{0} m^k
							\right ) = \sum_{i=1}^{m} \left (
									\binom{k}{2} b_i^{k-2} m^2 + \cdots
									\binom{k}{k} b_i^{0} m^k
							\right )
				\end{split}
				\end{equation}
				得證 $S(n+1)$ 滿足 $0 \le k < n+1$ 時,
				\begin{equation}
				\begin{split}
					\sum_{i=1}^{m} a_i^k + \sum_{i=1}^{m} c_i^k &= 
						\sum_{i=1}^{m} b_i^k + \sum_{i=1}^{m} d_i^k \\ 
				\end{split}
				\end{equation}
			\item 
		\end{itemize}
	\item 由數學歸納法得證,$S(n)$ 滿足 $\sum_{i=1}^{m} a_i^k = \sum_{i=1}^{m} b_i^k \text{, for all } 0 \le k < n$。
\end{itemize}


\end{document}