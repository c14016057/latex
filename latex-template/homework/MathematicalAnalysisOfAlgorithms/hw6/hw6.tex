%=======================02-713 LaTeX template, following the 15-210 template==================
%
% You don't need to use LaTeX or this template, but you must turn your homework in as
% a typeset PDF somehow.
%
% How to use:
%    1. Update your information in section "A" below
%    2. Write your answers in section "B" below. Precede answers for all 
%       parts of a question with the command "\question{n}{desc}" where n is
%       the question number and "desc" is a short, one-line description of 
%       the problem. There is no need to restate the problem.
%    3. If a question has multiple parts, precede the answer to part x with the
%       command "\part{x}".
%    4. If a problem asks you to design an algorithm, use the commands
%       \algorithm, \correctness, \runtime to precede your discussion of the 
%       description of the algorithm, its correctness, and its running time, respectively.
%    5. You can include graphics by using the command \includegraphics{FILENAME}
%
\documentclass[11pt]{article}

\usepackage[UTF8, heading = false, scheme = plain]{ctex} % chinese !!!!
\usepackage{amsmath,amssymb,amsthm}
\usepackage{graphicx}
\usepackage[margin=1in]{geometry}
\usepackage{fancyhdr}
\usepackage{CJK}
\usepackage{manfnt}
\usepackage{listings}

\usepackage{algorithm}% http://ctan.org/pkg/algorithm
\usepackage{algpseudocode}% http://ctan.org/pkg/algorithmicx


\setCJKmainfont{STHeiti} % chinese type

\setlength{\parindent}{0pt}
\setlength{\parskip}{5pt plus 1pt}
\setlength{\headheight}{13.6pt}

% define
\newcommand\question[2]{\vspace{.25in}\hrule\textbf{#1: #2}\vspace{.5em}\hrule\vspace{.10in}}
\renewcommand\part[1]{\vspace{.10in}\textbf{#1}}
\newcommand\correctness{\vspace{.10in}\textbf{Correctness: }}
\newcommand\runtime{\vspace{.10in}\textbf{Running time: }}
\pagestyle{fancyplain}

% header
\lhead{\textbf{\NAME\ (\ANDREWID)}}
\chead{\textbf{HW\HWNUM}}
\rhead{演算法數學分析, \today}


% start content
\begin{document}\raggedright
%Section A==============Change the values below to match your information==================
\newcommand\NAME{Shiang-Yun Yang 楊翔雲}  % your name
\newcommand\ANDREWID{R04922067}     % your andrew id
\newcommand\HWNUM{6}              % the homework number
%Section B==============Put your answers to the questions below here=======================

% no need to restate the problem --- the graders know which problem is which,
% but replacing "The First Problem" with a short phrase will help you remember
% which problem this is when you read over your homeworks to study.

\question{Problem 1} {6-12 Prove that Stirling numbers have an inversion law
	analogous to (5.48):
	\begin{align*}
		g(n) = \sum_{k} \begin{Bmatrix} n \\ k \end{Bmatrix} (-1)^{k} f(k)
		\Leftrightarrow 
		f(n) = \sum_{k} \begin{bmatrix} n \\ k \end{bmatrix} (-1)^{k} g(k)
	\end{align*}
}

\part{Answer:}

\begin{itemize}
	\item 證明 ($\Rightarrow$),如果 $g(n) = \sum_{k} \begin{Bmatrix} n \\ k \end{Bmatrix} (-1)^{k} f(k)$,則
		\begin{align*}
			\sum_{k} \begin{bmatrix} n \\ k \end{bmatrix} (-1)^{k} g(k) 
				&=	\sum_{k} \begin{bmatrix} n \\ k \end{bmatrix} 
					\sum_j \begin{Bmatrix} k \\ j \end{Bmatrix}(-1)^j f(j)\\
				&=	\sum_{j} f(j) \sum_{k} \begin{bmatrix} n \\ k \end{bmatrix}
					\begin{Bmatrix} k \\ j \end{Bmatrix} (-1)^{k+j} 
						&& \text{重新排列}\\
				&=	\sum_{j} f(j) \sum_{k} \begin{bmatrix} n \\ k \end{bmatrix}
					\begin{Bmatrix} k \\ j \end{Bmatrix} (-1)^{j-k} 
						&& \because \text{(6.31),} 
						\sum_{k} \begin{bmatrix} n \\ k \end{bmatrix}
						\begin{Bmatrix} k \\ m \end{Bmatrix} (-1)^{n-k} = [m = n] \\
				&= \sum_{j} f(j) [n = j]\\
				&= f(n)
		\end{align*}
	\item 同理 ($\Leftarrow$),如果 $f(n) = \sum_{k} \begin{bmatrix} n \\ k \end{bmatrix} (-1)^{k} g(k)$,則
		\begin{align*}
			\sum_{k} \begin{Bmatrix} n \\ k \end{Bmatrix} (-1)^{k} f(k)
				&=	\sum_{k} \begin{Bmatrix} n \\ k \end{Bmatrix} (-1)^{k} 
					\sum_{j} \begin{bmatrix} k \\ j \end{bmatrix} (-1)^{j} g(j) \\
				&=	\sum_{j} g(j) \sum_{k} \begin{Bmatrix} n \\ k \end{Bmatrix} 
					\begin{bmatrix} k \\ j \end{bmatrix} (-1)^{j-k} 
					&& \because \text{(6.31)} \\
				&= \sum_{j} g(j) [n = j] \\
				&= g(n)
		\end{align*}
\end{itemize}

\newpage

\question{Problem 2} {6-20 Find a closed form for $\sum_{k=1}^{n} H_k^{(2)}$
}

\part{Answer:}

\begin{align*}
\sum_{k=1}^{n} H_n^{(2)} 
	&= \sum_{1 \le k \le n} \sum_{1 \le j \le k} \frac{1}{j^2} \\
	&= \sum_{1 \le j \le k \le n} \frac{1}{j^2} \\
	&= \sum_{1 \le j \le n} \frac{1}{j^2} (n - j + 1) 
		&& \text{每一項} \frac{1}{j^2} \text{ 出現} n - j + 1 \text{ 次} \\
	&= \sum_{1 \le j \le n} \frac{n+1}{j^2} - \sum_{1 \le j \le n} \frac{1}{j} \\
	&= (n+1) H_n^{(2)} - H_n
\end{align*}

得 $\sum_{k=1}^{n} H_n^{(2)} = (n+1) H_n^{(2)} - H_n$。

\question{Problem 3} {6-39 Express $\sum_{k=1}^{n} H_k^2$ in terms of $n$ and $H_n$.
}

\part{Answer:}

\begin{align*}
\sum_{k=1}^{n} H_k^2 
	&= \sum_{1 \le k \le n} \sum_{1 \le j \le k} H_k \times \frac{1}{j} \\
	&= \sum_{1 \le j \le k \le n} \frac{1}{j} H_k \\
	&= \sum_{1 \le j \le n} \frac{1}{j} \sum_{j \le k \le n} H_k \\
	&= \sum_{1 \le j \le n} \frac{1}{j} \left ( (n+1) H_{n+1} - (n+1)
			- j H_j + j\right ) \\
	&= \left ( (n+1) H_{n+1} - (n+1) \right ) \sum_{1 \le j \le n} \frac{1}{j}
			- \sum_{1 \le j \le n} (H_j - 1) \\
	&= \left ( (n+1) H_{n+1} - (n+1) \right ) H_n
			- \left ( (n+1) H_{n+1} - (n+1) \right ) + n \\
	&= (n+1) H_{n+1} H_n - (n+1) H_n - (n+1) H_{n+1} + 2n - 1 \\
	&= ((n+1) H_{n} + 1) H_n - (n+1) H_n - ((n+1) H_{n} + 1) + 2n - 1 \\
	&= (n+1) H_{n}^2 - (2n+1) H_n + 2n
\end{align*}

得 $\sum_{k=1}^{n} H_k^2 = (n+1) H_{n}^2 - (2n+1) H_n + 2n$。

\newpage

\question{Problem 4} {5-49 Use the hypergeometric method to evaluate
	$$\sum_k (-1)^k \binom{x}{k} \binom{x+n-k}{n-k} \frac{y}{y+n-k}$$
}

\part{Answer:}

\begin{itemize}
	\item 代入 $k = 0$,得 $t_0 = \binom{x+n}{n} \frac{y}{y+n}$。
	\item 
		\begin{align*}
			\frac{t_{k+1}}{t_k} 
				&= \frac{(-1)^{k+1} \binom{x}{k+1} \binom{x+n-k-1}{n-k-1} \frac{y}{y+n-k-1}}{(-1)^{k} \binom{x}{k} \binom{x+n-k}{n-k} \frac{y}{y+n-k}} \\
				&= -1 \times \frac{\frac{x!}{(k+1)!(x-k-1)!}\frac{(x+n-k-1)!}{x!(n-k-1)!}\frac{1}{y+n-k-1}}{\frac{x!}{(k)!(x-k)!}\frac{(x+n-k)!}{x!(n-k)!}\frac{1}{y+n-k}} \\
				&= -1 \frac{(x-k)(n-k)(y+n-k)}{(k+1)(x+n-k)(y+n-k-1)} \\
				&= \frac{(k-x)(k-n)(k-(y+n))}{(k+1)(k-(x+n))(k-(y+n-1))} \\
		\end{align*}
	\item 
		\begin{align*}
			\text{原式}
				&= \binom{x+n}{n} \frac{y}{y+n} 
					F \left (\left.\begin{matrix} -x, -n, -y-n & \\  -x-n, -y-n+1 & 
								\end{matrix}\right|1 \right ) 
					&& \text{(5.97)}
						\begin{pmatrix}
							a, & b \\
							-n, & c 
						\end{pmatrix}\leftarrow
						\begin{pmatrix}
							-x, & -y-n \\
							-n, & -x-n
						\end{pmatrix} \\
				&= \binom{x+n}{n} \frac{y}{y+n} 
					\frac{(-x-n+x)^{\overline{n}} (-x-n+y+n)^{\overline{n}}}
					{(-x-n)^{\overline{n}} (-x-n+x+y+n)^{\overline{n}}} \\
				&= \binom{x+n}{n} \frac{y}{y+n}
						\frac{(-n)^{\overline{n}} (y-x)^{\overline{n}}}
						{(-x-n)^{\overline{n}} y^{\overline{n}}} \\
				&= \frac{(x+n)^{\underline{n}}}{n^{\underline{n}}} \frac{y}{y+n}
						\frac{(-1)^n n^{\underline{n}}}{(-1)^n (x+n)^{\underline{n}}}
						\frac{(y-x)^{\overline{n}}}{y^{\overline{n}}} \\
				&= \frac{y}{y+n} \frac{(y-x)^{\overline{n}}}{y^{\overline{n}}} \\
				&= \frac{(y-x)^{\overline{n}}}{(y+1)^{\overline{n}}}
		\end{align*}
\end{itemize}

\newpage 

\question{Problem 5} {This problem analyzes a hashing method called \textbf{uniform hashing}.
	\begin{align*}
		p_{nk} &= Pr\{Y_n = k\} \\
		p_{nk} &= Pr\{\text{slots } h_1(y), \cdots, h_{k-1}(y) \text{are occupied and slot } h_k(y) \text{ is not occupied} \}\\
		p_{nk} &= \frac{n^{\underline{k-1}}}{m^{\underline{k}}} (m-n)
	\end{align*}
	\begin{description}
		\item[a.] For $n < m$, compute $$E(Y_n) = \sum_{1 \le k \le n+1} k p_{nk}$$
		\item[b.] Compute $E(X_n)$ using part (a), $$E(X_n) = \frac{1}{n} \sum_{1 \le k \le n} E(Y_{k-1})$$
	\end{description}
}

\begin{description}
	\item[a.] $\sum_{1 \le k \le n+1} k p_{nk} = \sum_{1 \le k \le n+1} k\frac{n^{\underline{k-1}}}{m^{\underline{k}}} (m-n) = (m-n) \sum_{1 \le k \le n+1} k\frac{n^{\underline{k-1}}}{m^{\underline{k}}}$,單獨計算 $\sum_{1 \le k \le n+1} k\frac{n^{\underline{k-1}}}{m^{\underline{k}}}$。
		\begin{itemize}
			\item 使用超幾何函數找解,將 $\sum_{1 \le k \le n+1} k\frac{n^{\underline{k-1}}}{m^{\underline{k}}}$ 相當於 $\sum_{k \ge 0} (k+1)\frac{n^{\underline{k}}}{m^{\underline{k+1}}}$
			\item 當 $k = 0$ 代入,得到 $t_0 = \frac{1}{m}$。
			\item 求 $\frac{t_{k+1}}{t_k}$
				\begin{align*}
					\frac{t_{k+1}}{t_k} &=
						\frac{(k+2) \frac{n^{\underline{k+1}}}{m^{\underline{k+2}}}}{(k+1) \frac{n^{\underline{k}}}{m^{\underline{k+1}}}} \\
							&= \frac{(k+2)(n-k)}{(k+1)(m-k-1)} \\
							&= \frac{(k+2)(k-n)}{(k-m+1)(k+1)}
				\end{align*}
			\item 得超幾何表示 
				\begin{align*}
					\sum_{k \ge 0} (k+1)\frac{n^{\underline{k}}}{m^{\underline{k+1}}} 
						&= \frac{1}{m} F \left (\left.\begin{matrix} 2, -n & \\  -m+1 & 
								\end{matrix}\right|1 \right ) 
								&& \text{代入式 (5.93) 化簡}\\
						&= \frac{1}{m} \frac{(-m-1)^{\overline{n}}}{(-m+1)^{\overline{n}}} \\
						&= \frac{1}{m} \frac{(-m-1)(-m)}{(-m-1+n)(-m+n)} \\
						&= \frac{m+1}{(m-n+1)(m-n)}
				\end{align*}
		\end{itemize}
		
		最後將其代回原式
		\begin{align*}
			E(Y_n) &= \sum_{1 \le k \le n+1} k p_{nk} \\
					&= (m-n) \sum_{1 \le k \le n+1} k\frac{n^{\underline{k-1}}}{m^{\underline{k}}} \\
					&= (m-n) \frac{m+1}{(m-n+1)(m-n)} \\
					&= (m+1)/(m-n+1)
		\end{align*}
	\item[b.] 代入 part a 的解。
		\begin{align*}
			E(X_n) &= \frac{1}{n} \sum_{1 \le k \le n} E(Y_{k-1}) \\
				&= \frac{1}{n} \sum_{1 \le k \le n} \frac{m+1}{m-(k-1)+1} \\
				&= \frac{m+1}{n} \sum_{1 \le k \le n} \frac{1}{m-k+2} \\
				&= \frac{m+1}{n} (H_{m+1} - H_{m-n+1})
		\end{align*}
		
		得 $E(X_n) = \frac{m+1}{n} (H_{m+1} - H_{m-n+1})$。
\end{description}

\end{document}