%=======================02-713 LaTeX template, following the 15-210 template==================
%
% You don't need to use LaTeX or this template, but you must turn your homework in as
% a typeset PDF somehow.
%
% How to use:
%    1. Update your information in section "A" below
%    2. Write your answers in section "B" below. Precede answers for all 
%       parts of a question with the command "\question{n}{desc}" where n is
%       the question number and "desc" is a short, one-line description of 
%       the problem. There is no need to restate the problem.
%    3. If a question has multiple parts, precede the answer to part x with the
%       command "\part{x}".
%    4. If a problem asks you to design an algorithm, use the commands
%       \algorithm, \correctness, \runtime to precede your discussion of the 
%       description of the algorithm, its correctness, and its running time, respectively.
%    5. You can include graphics by using the command \includegraphics{FILENAME}
%
\documentclass[11pt]{article}

\usepackage[UTF8, heading = false, scheme = plain]{ctex} % chinese !!!!
\usepackage{amsmath,amssymb,amsthm}
\usepackage{graphicx}
\usepackage[margin=1in]{geometry}
\usepackage{fancyhdr}
\usepackage{CJK}
\usepackage{manfnt}
\usepackage{listings}


\setCJKmainfont{STHeiti} % chinese type

\setlength{\parindent}{0pt}
\setlength{\parskip}{5pt plus 1pt}
\setlength{\headheight}{13.6pt}

% define
\newcommand\question[2]{\vspace{.25in}\hrule\textbf{#1: #2}\vspace{.5em}\hrule\vspace{.10in}}
\renewcommand\part[1]{\vspace{.10in}\textbf{#1}}
\newcommand\algorithm{\vspace{.10in}\textbf{Algorithm: }}
\newcommand\correctness{\vspace{.10in}\textbf{Correctness: }}
\newcommand\runtime{\vspace{.10in}\textbf{Running time: }}
\pagestyle{fancyplain}

% header
\lhead{\textbf{\NAME\ (\ANDREWID)}}
\chead{\textbf{HW\HWNUM}}
\rhead{演算法數學解析, \today}


% start content
\begin{document}\raggedright
%Section A==============Change the values below to match your information==================
\newcommand\NAME{Shiang-Yun Yang 楊翔雲}  % your name
\newcommand\ANDREWID{R04922067}     % your andrew id
\newcommand\HWNUM{2}              % the homework number
%Section B==============Put your answers to the questions below here=======================

% no need to restate the problem --- the graders know which problem is which,
% but replacing "The First Problem" with a short phrase will help you remember
% which problem this is when you read over your homeworks to study.


\question{Problem 1}{$\mbox{\mancube}_{n} = \sum\nolimits_{k=1}^{n} k^3$ by the text's Method 5 as follows: First write $\mbox{\mancube}_{n} + \square_n = 2 \sum\nolimits_{1 \le j \le k \le n} jk$; then apply (2.33)
}

\part{Answer:}

\begin{enumerate}
	\item 第一步
		\begin{align*}
			\mbox{\mancube}_n + \square_n &= \sum_{1 \le k \le n} k^3 + \sum_{1 \le k \le n} k^2 && \text{合併}\\
				&= \sum_{1 \le k \le n} k^2 (k+1) && \text{分項} \\
				&= \sum_{1 \le k \le n} k \times k(k+1) && \because \sum_{1 \le j \le k} 2 j = k(k+1)\\
				&= \sum_{1 \le k \le n} k \sum_{1 \le j \le k} 2 j \\
				&= 2 \sum\nolimits_{1 \le j \le k \le n} jk
		\end{align*}
		
	\item 第二步套用課本 2.33 的寫法
		$$S = \sum_{1 \le j \le k \le n} a_j a_k = \frac{1}{2} \left ( \left ( \sum_{k=1}^{n} a_k \right )^2 + \sum_{k=1}^{n} a_k^2 \right )$$
		
	\item 令 $a_n = n$
		\begin{align*}
			\mbox{\mancube}_n + \square_n &= 2 \sum\nolimits_{1 \le j \le k \le n} jk \\
				&= 2 \times \frac{1}{2} \left ( \left ( \sum_{k=1}^{n} k \right )^2 + \sum_{k=1}^{n} k^2 \right ) \\
				&= \left ( \sum_{1 \le k \le n} k \right )^2 + \sum_{1 \le k \le n} k^2 
				= \left ( \sum_{1 \le k \le n} k \right )^2 + \square_n
		\end{align*}
	\item 得到 $\mbox{\mancube}_n = \left ( \sum_{1 \le k \le n} k \right )^2 = \left ( \frac{n(n+1)}{2}\right )^2$
\end{enumerate}

\question{Problem 2}{Evaluate the sums $S_n = \sum\nolimits_{k=0}^{n}(-1)^{n-k}$, $T_n = \sum\nolimits_{k=0}^{n}(-1)^{n-k} k$, and $U_n = \sum\nolimits_{k=0}^{n}(-1)^{n-k} k^2$ by the perturbation method, assuming that $n \ge 0$.
}

\part{Answer:}

\begin{itemize}
	\item 
		\begin{align*}
		S_{n+1} &= \sum_{k=0}^{n+1}(-1)^{(n+1)-k} && \text{移除末項}\\
				&= (-1)^0 + (-1)\sum_{k=0}^{n}(-1)^{n-k} = 1 - S_n \\
		S_{n+1} &= \sum_{k=0}^{n+1}(-1)^{(n+1)-k} && \text{移除首項}\\
				&= (-1)^{n+1} + \sum_{k=1}^{n+1}(-1)^{(n+1)-k} \\
				&= (-1)^{n+1} + \sum_{k=0}^{n}(-1)^{n-k} && k \leftarrow k+1 \\
				&= (-1)^{n+1} + S_n
		\end{align*}
		兩式相減得到 $0 = 1 - (-1)^{n+1} - 2 S_n$,最後得到 $S_n = \frac{1 - (-1)^{n+1}}{2} = \frac{1 + (-1)^n}{2} = [n \text{ is even}]$。
		
	\item 
		\begin{align*}
		T_{n+1} &= \sum_{k=0}^{n+1} (-1)^{n+1-k} k && \text{移除末項}\\
				&= (-1)^0(n+1) + (-1)\sum_{k=0}^{n} (-1)^{n-k} k = (n+1) - T_n \\
		T_{n+1} &= \sum_{k=0}^{n+1} (-1)^{n+1-k} k && \text{移除首項}\\
				&= 0 + \sum_{k=1}^{n+1} (-1)^{n+1-k} k \\
				&= \sum_{k=0}^{n} (-1)^{n-k} (k+1) && k \leftarrow k+1 \\
				&= \sum_{k=0}^{n} (-1)^{n-k} k + \sum_{k=0}^{n} (-1)^{n-k} = T_n + S_n
		\end{align*}
		兩式相減得到 $0 = (n+1) - 2 T_n - S_n$,最後得到 $T_n = \frac{n+1 - S_n}{2} = \frac{1}{2}(n + [n \text{ is odd}])$
	
	\item
		\begin{align*}
		U_{n+1} &= \sum_{k=0}^{n+1} (-1)^{n+1-k} k^2 && \text{移除末項}\\
				&= (n+1)^2 + (-1)\sum_{k=0}^{n} (-1)^{n-k} k^2 = (n+1)^2 - U_n \\
		U_{n+1} &= \sum_{k=0}^{n+1} (-1)^{n+1-k} k^2 && \text{移除首項}\\
				&= 0 + \sum_{k=1}^{n+1} (-1)^{n+1-k} k^2 \\
				&= \sum_{k=0}^{n} (-1)^{n-k} (k+1)^2 && k \leftarrow k+1 \\
				&= U_n + 2 T_n + S_n = U_n + n + [n \text{ is odd}] + [n \text{ is even}] \\
				&= U_n + n + 1
		\end{align*}
		兩式相減得到 $0 = (n+1)^2 - 2 U_n - n - 1$,最後得到 $U_n = \frac{1}{2} (n^2 + n)$。
\end{itemize}

\question{Problem 3}{2.24 What is $\sum\nolimits_{0 \le k < n} H_k \frac{1}{(k+1)(k+2)}$
	\textit{Hint:} Generalize the + of (2.57).
}

\part{Answer:}

\begin{align*}
	\sum_{0 \le k < n} H_k \times \frac{1}{(k+1)(k+2)} 
		&= \sum_{0 \le x < n} H_x \times \frac{1}{(x+1)(x+2)} \delta_x \\
		&= \sum_{0 \le x < n} H_x \cdot x^{\underline{-2}} \delta_x \\
\end{align*}

根據積分公式 $\sum u \Delta v = uv - \sum \mathrm{E} v \Delta u$,令 $u(x) = H_x$, $\Delta v(x) = x^{\underline{-2}} \delta_x$,則 $\Delta u(x) = x^{\underline{-1}}$, $v(x) = -1 \cdot x^{\underline{-1}}$,$\mathrm{E} v(x) = - \frac{1}{x+2}$,將原式改寫

\begin{align*}
	\sum_{0 \le x < n} H_x \cdot x^{\underline{-2}} \delta_x 
		&= H_x (-1) x^{\underline{-1}} - \sum_{0 \le x < n} \frac{-1}{x+2} \times \frac{1}{x+1} \delta_x \\
		&= - H_x x^{\underline{-1}} + \sum_{0 \le x < n} x^{\underline{-2}} \delta_x \\
		&= \left.\begin{matrix}- H_x x^{\underline{-1}} + (-1) x^{\underline{-1}}\end{matrix}\right|^n_0 \\
		&= \left.\begin{matrix}- H_x \frac{1}{x+1} - \frac{1}{x+1}\end{matrix}\right|^n_0 \\
		&= \left ( - H_n / (n+1) - 1 / (n+1) \right ) - \left ( - 1 / 2 - 1 / 2 \right ) \\
		&= 1 - \frac{H_n + 1}{n+1}
\end{align*}

得 $\sum\nolimits_{0 \le k < n} H_k \frac{1}{(k+1)(k+2)} = 1 - \frac{H_n + 1}{n+1}$

\question{Problem 4}{ Evaluate $\sum_{1 \le k \le 2^n - 1} \left \lfloor k \lg k \right \rfloor$
}

\part{Answer:}

\question{Problem 5}{Consider the expansion
	$$(1-a)(1-b)(1-c)(1-d) \cdots = 1 - a - b - ab - c + ac + bc - abc - d + \cdots $$
	What is the sign of the $n$-th term? (Hint: the answer has something to do with the number of ones in the binary expansion of $n$.)
}

\part{Answer:}

$n$ 的二進制表示法中,$n = \sum a_i 2^i, \; a_i \in \{0, 1\}$

$$
\begin{cases}
\text{positive} & \sum a_i \text{ is odd} \\
\text{negative} & \sum a_i \text{ is even}
\end{cases}
$$

\question{Problem 6}{3-40 The spiral function $\sigma(n)$, indicated in the diagram below, maps a non-negative integer $n$ onto an ordered pair of integers $(x(n), y(n))$. For example, it maps $n = 9$ onto the ordered pair $(1, 2)$. (a) Prove that if $m = \left \lfloor \sqrt{n} \right \rfloor$, $$x(n) = (-1)^m \left ( (n - m(m+1)) \cdot \left [ \left \lfloor 2 \sqrt{n} \text{ is even}\right \rfloor \right ] + \left \lceil \frac{1}{2} m \right \rceil \right )$$, and find a similar formula for $y(n)$. \textit{Hint:} Classify the spiral into segments $W_k, \; S_k, \; E_k, \; N_k$ according as $\left \lfloor 2 \sqrt{n} \right \rfloor = 4k - 2, 4k - 1, 4k, 4k+1$.
}

分成四個移動方向討論,如圖 $\overrightarrow{p_1 p_2} \in W_k$, $\overrightarrow{p_3 p_4} \in S_k$。

假設 $m = \left \lfloor \sqrt{n} \right \rfloor$,$\left \lfloor 2 \sqrt{n} \right \rfloor = 4k + r, \; -2 \le r \le 1$

\begin{itemize}
	\item 往西方向 $W_k$,得 $\left \lfloor 2 \sqrt{n} \right \rfloor = 4k - 2$ 和 $m = \left \lfloor \sqrt{n} \right \rfloor = 2k - 1$,且 $(2k - 1)^2\le n \le (2k-1)^2 + 2k-1$
		\begin{align*}
			\left \lfloor 2 \sqrt{n} \right \rfloor = 4k - 2 &
				\Rightarrow 4k-2 \le 2 \sqrt{n} < 4k - 1 \\
			& \Rightarrow 2k-1 \le \sqrt{n} < 2k - 1/2 \\
			& \Rightarrow \left \lfloor \sqrt{n} \right \rfloor = 2k-1 \\
		\end{align*}
	推得 
		\begin{align*}
			x(n) &= - (n - (m \times m) - k - 1) \\
				 &= m^2 - n + k - 1 \\
				 &= m^2 - n + k - 1 + m - (2 k - 1) \\
				 &= m(m+1) - n - k \\
			y(n) &= k
		\end{align*}
	\item 同理往南 $S_k$,$\left \lfloor 2 \sqrt{n} \right \rfloor = 4k - 1$ 和 $m = \left \lfloor\sqrt{n} \right \rfloor = 2k - 1$,且 $(2k-1)^2 + 2k \le n < (2k-1)^2 + 2k + 2k-1$
		\begin{align*}
			\left \lfloor 2 \sqrt{n} \right \rfloor = 4k - 1 &
				\Rightarrow 4k-1 \le 2 \sqrt{n} < 4k \\
			& \Rightarrow 2k-1/2 \le \sqrt{n} < 2k \\
			& \Rightarrow \left \lfloor \sqrt{n} \right \rfloor = 2k-1 \\
		\end{align*}
		推得 
		\begin{align*}
			x(n) &= k \\
			y(n) &= - (n - m \times m - m) + k && \text{去除中心正方形,得到位移後,從 } y = k \text{ 往下移動} \\
				 &= m(m+1) - n + k
		\end{align*}
	\item 同理往東 $E_k$,$\left \lfloor 2 \sqrt{n} \right \rfloor = 4k$ 和 $m = \left \lfloor\sqrt{n} \right \rfloor = 2k$,且 $(2k-1)^2 + 4k - 1 \le n \le (2k-1)^2 + 6k - 1$
		\begin{align*}
			x(n) &= -k + (n - (m-1)^2 - m - (m-1)) = n - m(m+1) + k \\
			y(n) &= -k
		\end{align*}
	\item 同理往北 $N_k$,$\left \lfloor 2 \sqrt{n} \right \rfloor = 4k+1$ 和 $m = \left \lfloor\sqrt{n} \right \rfloor = 2k$,且 $(2k-1)^2 + 6k - 1 < n < (2k-1)^2 + 8k$
		\begin{align*}
			x(n) &= k \\
			y(n) &= -k + (n - (m-1)^2 - m - (m-1) - m) = n - m(m+1) - k
		\end{align*}
	\item 代入四個方向,得證 $$x(n) = (-1)^m \left ( (n - m(m+1)) \cdot \left [ \left \lfloor 2 \sqrt{n} \text{ is even}\right \rfloor \right ] + \left \lceil \frac{1}{2} m \right \rceil \right )$$
	\item 從推算每一個方向 $y(n)$ 中,方程式相似於 $x(n)$,得到 $$y(n) = (-1)^m \left ( (n - m(m+1)) \cdot \left [ \left \lfloor 2 \sqrt{n} \text{ is odd}\right \rfloor \right ] - \left \lceil \frac{1}{2} m \right \rceil \right )$$
\end{itemize}

\end{document}