\section{Implementation}

\subsection{Cooperative RK}
\begin{frame}
	\frametitle{Cooperative RK}
	\begin{itemize}
		\setlength\itemsep{1em}
		\item Use the scan operation provided by the Thrust library.
		\item All intermediate results are inthe form of arrays 2-by-2 
		integer matrices: $\mathcal{L}$, $\mathcal{A}$, $\mathcal{S}$,
		$\mathcal{T}$ each requires 16B per input character.
	\end{itemize}
\end{frame}

\subsection{Divide-and-Conquer RK}
\begin{frame}
	\frametitle{Divide-and-Conquer RK}
	\begin{itemize}
		\setlength\itemsep{1em}
		\item For binary patterns of size less than or equal 32 
		characters, we can use 32-bit registers.
		\item Using \texttt{SHIFT} and \texttt{AND} to replace module operators.
		(DRK without module operator: DRK-WOM)
		\item We read elements as \texttt{char4} vectors to better
		exploit the available memory bandwidth.
	\end{itemize}
\end{frame}
